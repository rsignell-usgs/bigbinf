% v2-acmsmall-sample.tex, dated March 6 2012
% This is a sample file for ACM small trim journals
%
% Compilation using 'acmsmall.cls' - version 1.3 (March 2012), Aptara Inc.
% (c) 2010 Association for Computing Machinery (ACM)
%
% Questions/Suggestions/Feedback should be addressed to => "acmtexsupport@aptaracorp.com".
% Users can also go through the FAQs available on the journal's submission webpage.
%
% Steps to compile: latex, bibtex, latex latex
%
% For tracking purposes => this is v1.3 - March 2012

\documentclass[prodmode,acmtecs]{acmsmall} % Aptara syntax
\usepackage{caption}
\usepackage{multicol}
\usepackage{float}
\usepackage{rotating}
\renewcommand{\arraystretch}{1.5}

% Document starts
\begin{document}

% Title portion
\title{Background Chapter}
\author{Andrew van Rooyen}

\acmformat{van Rooyen. Andrew, 2015. Background chapter}
% At a minimum you need to supply the author names, year and a title.
% IMPORTANT:
% Full first names whenever they are known, surname last, followed by a period.
% In the case of two authors, 'and' is placed between them.
% In the case of three or more authors, the serial comma is used, that is, all author names
% except the last one but including the penultimate author's name are followed by a comma,
% and then 'and' is placed before the final author's name.
% If only first and middle initials are known, then each initial
% is followed by a period and they are separated by a space.
% The remaining information (journal title, volume, article number, date, etc.) is 'auto-generated'.

\maketitle

\section*{2. Background}
Data sets are a big part of bioinformatics, and have introduced many new challenges with the rise of next generation sequencing. Sequencing technologies like SOLiD provide much higher data output at a cheaper cost~\cite{shendure2008next}, which is good news for research, but troubling for data storage, transfer and access. In fact, the cost of storing a byte has been more expensive than sequencing a base pair since before 2010~\cite{baker2010next}.

This makes it difficult for researchers in different locations to manipulate and run processes on the data, because it will be stored in only one location. These files could be tens of gigabytes in size~\cite{deorowicz2011compression}, depending on context.

\section*{2.1 Data storage}
Generally, once the sequencing machine has generated the raw information on each base pair, this data will be stored in a data warehouse. Storing this information for long periods of time requires the data to be structured efficiently in order to save space, and allow it to be transferred efficiently.
There has been a lot of work on how to structure this data. There are a plethora of file formats whose efficiency depends on the kind of data which needs to be stored. Two of the most popular formats are FASTQ, which stores aggregated reads along with the quality of each base pair~\cite{cock2010sanger}, and BAM, the binary, compressed version of the Sequence Alignment Map (SAM) format~\cite{SAMspec}.


\section*{2.2 Data transfer}
When researchers require specific information for their projects, they need to be able to access the data warehouse and transfer whichever sequences they need. Luckily, these locations are often connected massive data pipes like National Research and Education Networks (NRENs). Unfortunately, standard protocols like FTP and SSH were never designed for use on high-throughput networks, and alternate protocols need to be used to avoid bottlenecking.

There are some proprietary transfer protocols which are widely used in practice. For example, the fasp protocol by the US based company AsperaSoft. Based on UDP, the protocol eliminates the latency issues seen with TCP, and provides bandwidth up to 10 gigabits per second to transfer data~\cite{beloslyudtsev2014aspera}.

\section*{2.3 Alternate models}
There have been some attempts to do data processing remotely, and there has been an explorative push towards cloud solutions from Amazon, Google etc. Unfortunately, even though these cloud data centres have plenty of cheap storage, there are very significant drawbacks.

Because the sequencing happens in labs, researchers need to upload their raw data to the cloud data centres every time they run a new experiment. This leads back to the original problem, as researchers resort to mailing hard drives~\cite{baker2010next}.
There are also security, privacy and ethical concerns with outsourcing this processing power to other companies, as sequenced DNA data is often highly sensitive information~\cite{marx2013biology}.

% Bibliography
\bibliographystyle{ACM-Reference-Format-Journals}
\bibliography{ref}
\end{document}
% End of v2-acmsmall-sample.tex (March 2012) - Gerry Murray, ACM

